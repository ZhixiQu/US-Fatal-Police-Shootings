% Options for packages loaded elsewhere
% Options for packages loaded elsewhere
\PassOptionsToPackage{unicode}{hyperref}
\PassOptionsToPackage{hyphens}{url}
\PassOptionsToPackage{dvipsnames,svgnames,x11names}{xcolor}
%
\documentclass[
  letterpaper,
  DIV=11,
  numbers=noendperiod]{scrartcl}
\usepackage{xcolor}
\usepackage{amsmath,amssymb}
\setcounter{secnumdepth}{5}
\usepackage{iftex}
\ifPDFTeX
  \usepackage[T1]{fontenc}
  \usepackage[utf8]{inputenc}
  \usepackage{textcomp} % provide euro and other symbols
\else % if luatex or xetex
  \usepackage{unicode-math} % this also loads fontspec
  \defaultfontfeatures{Scale=MatchLowercase}
  \defaultfontfeatures[\rmfamily]{Ligatures=TeX,Scale=1}
\fi
\usepackage{lmodern}
\ifPDFTeX\else
  % xetex/luatex font selection
\fi
% Use upquote if available, for straight quotes in verbatim environments
\IfFileExists{upquote.sty}{\usepackage{upquote}}{}
\IfFileExists{microtype.sty}{% use microtype if available
  \usepackage[]{microtype}
  \UseMicrotypeSet[protrusion]{basicmath} % disable protrusion for tt fonts
}{}
\makeatletter
\@ifundefined{KOMAClassName}{% if non-KOMA class
  \IfFileExists{parskip.sty}{%
    \usepackage{parskip}
  }{% else
    \setlength{\parindent}{0pt}
    \setlength{\parskip}{6pt plus 2pt minus 1pt}}
}{% if KOMA class
  \KOMAoptions{parskip=half}}
\makeatother
% Make \paragraph and \subparagraph free-standing
\makeatletter
\ifx\paragraph\undefined\else
  \let\oldparagraph\paragraph
  \renewcommand{\paragraph}{
    \@ifstar
      \xxxParagraphStar
      \xxxParagraphNoStar
  }
  \newcommand{\xxxParagraphStar}[1]{\oldparagraph*{#1}\mbox{}}
  \newcommand{\xxxParagraphNoStar}[1]{\oldparagraph{#1}\mbox{}}
\fi
\ifx\subparagraph\undefined\else
  \let\oldsubparagraph\subparagraph
  \renewcommand{\subparagraph}{
    \@ifstar
      \xxxSubParagraphStar
      \xxxSubParagraphNoStar
  }
  \newcommand{\xxxSubParagraphStar}[1]{\oldsubparagraph*{#1}\mbox{}}
  \newcommand{\xxxSubParagraphNoStar}[1]{\oldsubparagraph{#1}\mbox{}}
\fi
\makeatother


\usepackage{longtable,booktabs,array}
\usepackage{calc} % for calculating minipage widths
% Correct order of tables after \paragraph or \subparagraph
\usepackage{etoolbox}
\makeatletter
\patchcmd\longtable{\par}{\if@noskipsec\mbox{}\fi\par}{}{}
\makeatother
% Allow footnotes in longtable head/foot
\IfFileExists{footnotehyper.sty}{\usepackage{footnotehyper}}{\usepackage{footnote}}
\makesavenoteenv{longtable}
\usepackage{graphicx}
\makeatletter
\newsavebox\pandoc@box
\newcommand*\pandocbounded[1]{% scales image to fit in text height/width
  \sbox\pandoc@box{#1}%
  \Gscale@div\@tempa{\textheight}{\dimexpr\ht\pandoc@box+\dp\pandoc@box\relax}%
  \Gscale@div\@tempb{\linewidth}{\wd\pandoc@box}%
  \ifdim\@tempb\p@<\@tempa\p@\let\@tempa\@tempb\fi% select the smaller of both
  \ifdim\@tempa\p@<\p@\scalebox{\@tempa}{\usebox\pandoc@box}%
  \else\usebox{\pandoc@box}%
  \fi%
}
% Set default figure placement to htbp
\def\fps@figure{htbp}
\makeatother


% definitions for citeproc citations
\NewDocumentCommand\citeproctext{}{}
\NewDocumentCommand\citeproc{mm}{%
  \begingroup\def\citeproctext{#2}\cite{#1}\endgroup}
\makeatletter
 % allow citations to break across lines
 \let\@cite@ofmt\@firstofone
 % avoid brackets around text for \cite:
 \def\@biblabel#1{}
 \def\@cite#1#2{{#1\if@tempswa , #2\fi}}
\makeatother
\newlength{\cslhangindent}
\setlength{\cslhangindent}{1.5em}
\newlength{\csllabelwidth}
\setlength{\csllabelwidth}{3em}
\newenvironment{CSLReferences}[2] % #1 hanging-indent, #2 entry-spacing
 {\begin{list}{}{%
  \setlength{\itemindent}{0pt}
  \setlength{\leftmargin}{0pt}
  \setlength{\parsep}{0pt}
  % turn on hanging indent if param 1 is 1
  \ifodd #1
   \setlength{\leftmargin}{\cslhangindent}
   \setlength{\itemindent}{-1\cslhangindent}
  \fi
  % set entry spacing
  \setlength{\itemsep}{#2\baselineskip}}}
 {\end{list}}
\usepackage{calc}
\newcommand{\CSLBlock}[1]{\hfill\break\parbox[t]{\linewidth}{\strut\ignorespaces#1\strut}}
\newcommand{\CSLLeftMargin}[1]{\parbox[t]{\csllabelwidth}{\strut#1\strut}}
\newcommand{\CSLRightInline}[1]{\parbox[t]{\linewidth - \csllabelwidth}{\strut#1\strut}}
\newcommand{\CSLIndent}[1]{\hspace{\cslhangindent}#1}



\setlength{\emergencystretch}{3em} % prevent overfull lines

\providecommand{\tightlist}{%
  \setlength{\itemsep}{0pt}\setlength{\parskip}{0pt}}



 


\KOMAoption{captions}{tableheading}
\makeatletter
\@ifpackageloaded{caption}{}{\usepackage{caption}}
\AtBeginDocument{%
\ifdefined\contentsname
  \renewcommand*\contentsname{Table of contents}
\else
  \newcommand\contentsname{Table of contents}
\fi
\ifdefined\listfigurename
  \renewcommand*\listfigurename{List of Figures}
\else
  \newcommand\listfigurename{List of Figures}
\fi
\ifdefined\listtablename
  \renewcommand*\listtablename{List of Tables}
\else
  \newcommand\listtablename{List of Tables}
\fi
\ifdefined\figurename
  \renewcommand*\figurename{Figure}
\else
  \newcommand\figurename{Figure}
\fi
\ifdefined\tablename
  \renewcommand*\tablename{Table}
\else
  \newcommand\tablename{Table}
\fi
}
\@ifpackageloaded{float}{}{\usepackage{float}}
\floatstyle{ruled}
\@ifundefined{c@chapter}{\newfloat{codelisting}{h}{lop}}{\newfloat{codelisting}{h}{lop}[chapter]}
\floatname{codelisting}{Listing}
\newcommand*\listoflistings{\listof{codelisting}{List of Listings}}
\makeatother
\makeatletter
\makeatother
\makeatletter
\@ifpackageloaded{caption}{}{\usepackage{caption}}
\@ifpackageloaded{subcaption}{}{\usepackage{subcaption}}
\makeatother
\usepackage{bookmark}
\IfFileExists{xurl.sty}{\usepackage{xurl}}{} % add URL line breaks if available
\urlstyle{same}
\hypersetup{
  pdftitle={My title},
  pdfauthor={First author; Another author},
  colorlinks=true,
  linkcolor={blue},
  filecolor={Maroon},
  citecolor={Blue},
  urlcolor={Blue},
  pdfcreator={LaTeX via pandoc}}


\title{My title\thanks{Code and data are available at:
\url{https://github.com/RohanAlexander/starter_folder}.}}
\usepackage{etoolbox}
\makeatletter
\providecommand{\subtitle}[1]{% add subtitle to \maketitle
  \apptocmd{\@title}{\par {\large #1 \par}}{}{}
}
\makeatother
\subtitle{My subtitle if needed}
\author{First author \and Another author}
\date{November 24, 2025}
\begin{document}
\maketitle
\begin{abstract}
First sentence. Second sentence. Third sentence. Fourth sentence.
\end{abstract}


\section{Introduction}\label{introduction}

Overview paragraph

Estimand paragraph

Results paragraph

Why it matters paragraph

Telegraphing paragraph: The remainder of this paper is structured as
follows. Section~\ref{sec-data}\ldots.

\section{Data}\label{sec-data}

\subsection{Overview}\label{overview}

We use the statistical programming language R (R Core Team 2023)\ldots.
Our data (Toronto Shelter \& Support Services 2024)\ldots. Following
Alexander (2023), we consider\ldots{}

Overview text

\subsection{Measurement}\label{measurement}

Some paragraphs about how we go from a phenomena in the world to an
entry in the dataset.

\subsection{Outcome variables}\label{outcome-variables}

Add graphs, tables and text. Use sub-sub-headings for each outcome
variable or update the subheading to be singular.

Some of our data is of penguins (Figure~\ref{fig-bills}), from Horst,
Hill, and Gorman (2020).

\begin{figure}

\centering{

\pandocbounded{\includegraphics[keepaspectratio]{paper_files/figure-pdf/fig-bills-1.pdf}}

}

\caption{\label{fig-bills}Bills of penguins}

\end{figure}%

Talk more about it.

And also planes (Figure~\ref{fig-planes}). (You can change the height
and width, but don't worry about doing that until you have finished
every other aspect of the paper - Quarto will try to make it look nice
and the defaults usually work well once you have enough text.)

\begin{figure}

\centering{

\pandocbounded{\includegraphics[keepaspectratio]{paper_files/figure-pdf/fig-planes-1.pdf}}

}

\caption{\label{fig-planes}Relationship between wing length and width}

\end{figure}%

Talk way more about it.

\subsection{Predictor variables}\label{predictor-variables}

Add graphs, tables and text.

Use sub-sub-headings for each outcome variable and feel free to combine
a few into one if they go together naturally.

\section{Model}\label{model}

The goal of our modelling strategy is twofold. Firstly,\ldots{}

Here we briefly describe the Bayesian analysis model used to
investigate\ldots{} Background details and diagnostics are included in
Appendix~\ref{sec-model-details}.

\subsection{Model set-up}\label{model-set-up}

Define \(y_i\) as the number of seconds that the plane remained aloft.
Then \(\beta_i\) is the wing width and \(\gamma_i\) is the wing length,
both measured in millimeters.

\begin{align} 
y_i|\p_i &\sim \mbox{Bernolli}(p_i) \\
\log\left(\frac{\hat{p}_i}{1-\hat{p}_i}\right) &= \beta_0 + \beta_1 \,\text{race}_i + \beta_2 \,\text{gender}_i + \beta_3 \,\text{age}_i + \beta_4 \,\text{armed}_i + \beta_5 \,\text{flee}_i + \beta_6 \,\text{mental health}_i + \beta_7 \,\text{year}_i\\
\alpha &\sim \mbox{Normal}(0, 2.5) \\
\beta &\sim \mbox{Normal}(0, 2.5) \\
\gamma &\sim \mbox{Normal}(0, 2.5) \\
\sigma &\sim \mbox{Exponential}(1)
\end{align}

We run the model in R (R Core Team 2023) using the \texttt{rstanarm}
package of Goodrich et al. (2022). We use the default priors from
\texttt{rstanarm}.

\subsubsection{Model justification}\label{model-justification}

We expect a positive relationship between the size of the wings and time
spent aloft. In particular\ldots{}

We can use maths by including latex between dollar signs, for instance
\(\theta\).

\section{Results}\label{results}

Our results are summarized in Table~\ref{tbl-modelresults}.

\begin{table}

\caption{\label{tbl-modelresults}Explanatory models of flight time based
on wing width and wing length}

\centering{

\centering
\begin{tblr}[         %% tabularray outer open
]                     %% tabularray outer close
{                     %% tabularray inner open
colspec={Q[]Q[]},
hline{2}={1-2}{solid, black, 0.05em},
hline{8}={1-2}{solid, black, 0.05em},
hline{1}={1-2}{solid, black, 0.1em},
hline{18}={1-2}{solid, black, 0.1em},
column{2}={}{halign=c},
column{1}={}{halign=l},
}                     %% tabularray inner close
& First model \\
(Intercept) & \num{1.12} \\
& (\num{1.70}) \\
length & \num{0.01} \\
& (\num{0.01}) \\
width & \num{-0.01} \\
& (\num{0.02}) \\
Num.Obs. & \num{19} \\
R2 & \num{0.320} \\
R2 Adj. & \num{0.019} \\
Log.Lik. & \num{-18.128} \\
ELPD & \num{-21.6} \\
ELPD s.e. & \num{2.1} \\
LOOIC & \num{43.2} \\
LOOIC s.e. & \num{4.3} \\
WAIC & \num{42.7} \\
RMSE & \num{0.60} \\
\end{tblr}

}

\end{table}%

\section{Discussion}\label{discussion}

\subsection{First discussion point}\label{sec-first-point}

If my paper were 10 pages, then should be be at least 2.5 pages. The
discussion is a chance to show off what you know and what you learnt
from all this.

\subsection{Second discussion point}\label{second-discussion-point}

Please don't use these as sub-heading labels - change them to be what
your point actually is.

\subsection{Third discussion point}\label{third-discussion-point}

\subsection{Weaknesses and next steps}\label{weaknesses-and-next-steps}

Weaknesses and next steps should also be included.

\newpage

\appendix

\section*{Appendix}\label{appendix}
\addcontentsline{toc}{section}{Appendix}

\section{Additional data details}\label{additional-data-details}

\section{Model details}\label{sec-model-details}

\subsection{Posterior predictive
check}\label{posterior-predictive-check}

In Figure~\ref{fig-ppcheckandposteriorvsprior-1} we implement a
posterior predictive check. This shows\ldots{}

In Figure~\ref{fig-ppcheckandposteriorvsprior-2} we compare the
posterior with the prior. This shows\ldots{}

\begin{figure}

\begin{minipage}{0.50\linewidth}

\centering{

\pandocbounded{\includegraphics[keepaspectratio]{paper_files/figure-pdf/fig-ppcheckandposteriorvsprior-1.pdf}}

}

\subcaption{\label{fig-ppcheckandposteriorvsprior-1}Posterior prediction
check}

\end{minipage}%
%
\begin{minipage}{0.50\linewidth}

\centering{

\pandocbounded{\includegraphics[keepaspectratio]{paper_files/figure-pdf/fig-ppcheckandposteriorvsprior-2.pdf}}

}

\subcaption{\label{fig-ppcheckandposteriorvsprior-2}Comparing the
posterior with the prior}

\end{minipage}%

\caption{\label{fig-ppcheckandposteriorvsprior}Examining how the model
fits, and is affected by, the data}

\end{figure}%

\subsection{Diagnostics}\label{diagnostics}

Figure~\ref{fig-stanareyouokay-1} is a trace plot. It shows\ldots{} This
suggests\ldots{}

Figure~\ref{fig-stanareyouokay-2} is a Rhat plot. It shows\ldots{} This
suggests\ldots{}

\begin{figure}

\begin{minipage}{0.50\linewidth}

\centering{

\pandocbounded{\includegraphics[keepaspectratio]{paper_files/figure-pdf/fig-stanareyouokay-1.pdf}}

}

\subcaption{\label{fig-stanareyouokay-1}Trace plot}

\end{minipage}%
%
\begin{minipage}{0.50\linewidth}

\centering{

\pandocbounded{\includegraphics[keepaspectratio]{paper_files/figure-pdf/fig-stanareyouokay-2.pdf}}

}

\subcaption{\label{fig-stanareyouokay-2}Rhat plot}

\end{minipage}%

\caption{\label{fig-stanareyouokay}Checking the convergence of the MCMC
algorithm}

\end{figure}%

\newpage

\section*{References}\label{references}
\addcontentsline{toc}{section}{References}

\phantomsection\label{refs}
\begin{CSLReferences}{1}{0}
\bibitem[\citeproctext]{ref-tellingstories}
Alexander, Rohan. 2023. \emph{Telling Stories with Data}. Chapman;
Hall/CRC. \url{https://tellingstorieswithdata.com/}.

\bibitem[\citeproctext]{ref-rstanarm}
Goodrich, Ben, Jonah Gabry, Imad Ali, and Sam Brilleman. 2022.
{``{rstanarm: {Bayesian} applied regression modeling via {Stan}}.''}
\url{https://mc-stan.org/rstanarm/}.

\bibitem[\citeproctext]{ref-palmerpenguins}
Horst, Allison Marie, Alison Presmanes Hill, and Kristen B Gorman. 2020.
\emph{{palmerpenguins: Palmer Archipelago (Antarctica) penguin data}}.
\url{https://doi.org/10.5281/zenodo.3960218}.

\bibitem[\citeproctext]{ref-citeR}
R Core Team. 2023. \emph{{R: A Language and Environment for Statistical
Computing}}. Vienna, Austria: R Foundation for Statistical Computing.
\url{https://www.R-project.org/}.

\bibitem[\citeproctext]{ref-shelter}
Toronto Shelter \& Support Services. 2024. \emph{Deaths of Shelter
Residents}.
\url{https://open.toronto.ca/dataset/deaths-of-shelter-residents/}.

\end{CSLReferences}




\end{document}
